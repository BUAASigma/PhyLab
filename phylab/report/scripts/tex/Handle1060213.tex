\subsection*{实验2.自准直法测凸透镜焦距}

\begin{center}

\begin{tabular}{|c|c|c|c|}
\hline 
光源/mm &  凸透镜1/mm & 凸透镜2/mm & 均值/mm \\ 
\hline 



%% i %%

& %% i %%


\\
\hline


\end{tabular}
\vspace{10pt}

\end{center}
$$\because f = {x}_{\text{光源}} - {x}_{\text{凸}} - \delta$$
$$\delta =  %% DELTA %% mm$$
$$\therefore {f}_1 = %% F[0] %% mm $$   
\begin{center}
${f}_2 = %% F[1] %% mm $   
${f}_3 = %% F[2] %% mm $ 
${f}_4 = %% F[3] %% mm $   
${f}_5 = %% F[4] %% mm $
\end{center}

$$\therefore \bar{f} = \frac{{f}_1+{f}_2+{f}_3+{f}_4+{f}_5}{5}  = %% AVERAGE_F %% mm$$
不确定度计算: 
A类不确定度 $${u}_a = \sqrt{\frac{\overline{x^{2}} - \bar{x} ^{2}}{k-1}} =  %% UA_F %% mm$$
B类不确定度 $${u}_b = \frac{0.5}{\sqrt{3}} = %% UB_F %% mm$$
f的不确定度
$$u(f) = \sqrt{{u}_a ^ {2} + {u}_b ^ {2}} =  %% UF %% mm$$
$$  {\therefore}   \text{最终结果} f = (%% FINAL %% ) mm $$