\section*{五、数据处理}

\subsection*{(一)测定钠光波长差}

%%x_aver%%
\subsubsection*{1、原始lalal数据列表} %原始数据
\begin{center}
\begin{tabular}{|c|c|c|c|c|c|c|c|c|c|c|}
\hline
i&1&2&3&4&5&6&7&8&9&10\\
\hline
$d_i \setminus mm $ & %% m1 %%  $ \\ %需要数组名称m/kg
\hline
\end{tabular}
\vspace{10pt}
\end {center}
\subsubsection*{2、数据处理——用一元线性回归法计算}
由原理可知:
${{\rm{d}}_{\rm{i}}}{\rm{ = }}\frac{{{{\bar \lambda }^{\rm{2}}}}}{{{\rm{2}}\Delta \lambda }}{\rm{i + }}{{\rm{d}}_{\rm{0}}}  $
令 $i\equiv x,d_i\equiv y$ 则
${\rm{b=}}\frac{{{{\bar \lambda }^{\rm{2}}}}}{{{\rm{2}}\Delta \lambda }}$

\begin{center}
\begin{tabular}{|c|c|c|c|c|c|c|c|c|c|c|c|}
\hline
\backslashbox{项目}{i}&1&2&3&4&5&6&7&8&9&10&平均\\
\hline
$x_i$&1&2&3&4&5&6&7&8&9&10&5.5\\
\hline
$y_i\setminus mm$& %% yi %%  $ & %%y_aver%%\\% 数组yi
\hline
$x_i^2$&1&4&9&16&25&36&49&64&81&100&38.5\\
\hline
$y_i^2\setminus mm^2$&%% yi2 %%  $ & %%y2_aver%% \\% 数组yi^2
\hline
$x_iy_i\setminus mm^2$&%% xyi %%  $ & %%xy_aver%%\\% 数组xi*yi
\hline
\end{tabular}
\vspace{10pt}
\end{center}

而$\bar \lambda  = \rm{ %% Lambda %% :}$
\[\therefore {\rm{b = }}\frac{{{{\bar \lambda }^{\rm{2}}}}}{{{\rm{2}}\Delta \lambda }}{\rm{ = }}\frac{{\overline {{\rm{xy}}}  - \bar x\bar y}}{{\overline {{{\rm{x}}^{\rm{2}}}} {\rm{ - }}{{\bar x}^{\rm{2}}}}}{\rm{ = %%b%% }}\] %b的计算

\[\therefore \Delta \lambda {\rm{ = }}\frac{{{{\bar \lambda }^{\rm{2}}}}}{{{\rm{2b}}}}{\rm{ = %%delta_lambda%%}}\] %△λ计算

相关系数:
\[{\rm{r = }}\frac{{\overline {{\rm{xy}}}  - \bar x\bar y}}{{\sqrt {{\rm{(}}{{\bar x}^{\rm{2}}}{\rm{ - }}\overline {{{\rm{x}}^{\rm{2}}}} {\rm{)}}{\rm{(}}{{\overline {\rm{y}} }^{\rm{2}}}{\rm{ - }}\overline {{{\rm{y}}^{\rm{2}}}}{\rm{)}} } }}{\rm{ = %%r%%}}\] %r的计算

$\therefore$ 有强烈的线性相关性

\subsubsection*{3、不确定度计算}
略去其他不确定度分量的贡献,
\[u\left( b \right) = b\sqrt {\frac{1}{{10 - 2}}\left( {\frac{1}{{{r^2}}} - 1} \right)} = %% u_b %%\] %u(b)
\[\because \ln b = \ln \frac{{{{\bar \lambda }^{\rm{2}}}}}{2} - \ln \Delta \lambda \]
\[\therefore u\left( {\Delta \lambda } \right) = \Delta \lambda \frac{{u\left( b \right)}}{b} = %%u_delta_lambda%%\] %u(△λ)

最终表述为:$\Delta \lambda  \pm u\left( {\Delta \lambda } \right) = (%%delta_lambda%% \pm %%u_delta_lambda%%)nm$

\subsubsection*{4、计算相对误差}
理论值:$\Delta\lambda=0.6nm$ %△λ
\par
相对误差:$\sigma=%%delta%%\%$  %σ

\subsection*{(二)用一元线性回归法验证常数}
\subsubsection*{1、原始数据表}

\begin{center}
\begin{tabular}{|c|c|c|c|c|c|c|c|c|}
\hline
i&1&2&3&4&5&6&7&8\\
\hline
$d_l\setminus mm$ & %% d_li %% \\
\hline
$d_r\setminus mm$ & %% d_ri %% \\
\hline
$D_i\setminus mm$ & %% Di %% \\
\hline
$D_i^2\setminus mm^2$ & %% D2i %% \\
\hline
\end{tabular}
\vspace{10pt}
\end{center}

\subsubsection*{2、数据处理}
由$D_k^2-D_{k+1}^2=\frac{4\lambda f^2}{nd}$ ,令  %这里公式没有问题吧
$k=i+k$,有$D_i^2=\frac{4\lambda f^2}{nd}i+\Delta$
\par
令$y=D_i^2$,$x=i$,有$y=a+bx$,$b=\frac{4\lambda f^2}{nd}$
\par
利用一元线性回归法可以处理得到结果:
\[\therefore {\rm{b = }}\frac{{{{\bar \lambda }^{\rm{2}}}}}{{{\rm{2}}\Delta \lambda }}{\rm{ = }}\frac{{\overline {{\rm{xy}}}  - \bar x\bar y}}{{\overline {{{\rm{x}}^{\rm{2}}}} {\rm{ - }}{{\bar x}^{\rm{2}}}}}{\rm{ =%%bb%% }}\] %b的计算

\[{\rm{r = }}\frac{{\overline {{\rm{xy}}}  - \bar x\bar y}}{{\sqrt {{\rm{(}}{{\bar x}^{\rm{2}}}{\rm{ - }}\overline {{{\rm{x}}^{\rm{2}}}} {\rm{)}}{\rm{(}}{{\overline {\rm{y}} }^{\rm{2}}}{\rm{ - }}\overline {{{\rm{y}}^{\rm{2}}}}{\rm{)}} } }}{\rm{ = %%rr%%}}\] %r的计算
$\therefore$线性关系强烈

$$d=\frac{4\lambda f^2}{n|b|}=%% d %% m$$
又因为$|r|\approx 1$ 知i与$D_i^2$线性关系强烈,由此验证$D_{i+1}^2-D_i^2=C$