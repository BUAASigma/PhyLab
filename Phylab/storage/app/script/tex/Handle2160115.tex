\documentclass[11pt,a4paper,oneside]{article}
\usepackage[UTF8]{ctex}
\usepackage{wrapfig}
\usepackage{indentfirst}
\usepackage{amsmath}
\usepackage{float}
\usepackage{ulem}
\usepackage[top=1in,bottom=1in,left=1.25in,right=1.25in]{geometry}
\usepackage{color}
\usepackage{xcolor}
\usepackage{multirow}
\usepackage{amssymb}
\usepackage{graphicx}
\usepackage{CJK}
\usepackage{slashbox} %添加了一个宏包

\begin{document}
\begin{CJK*}{GBK}{song}
\section*{五、数据处理}
$\rho=981kg/m^3,g=9.792m/s^2,\eta=1.83\times10^{-5}kg/m\cdot s,$
$b=8.22\times 10^{-3}m\cdot pa $
\par $P_{20}=1.0133\times 10^{-5}Pa,d=5.00\times 10^{-3}m,\theta =1.6021733\times 10^{-19}C$
\subsubsection*{1、原始数据列表} %原始数据
\begin{center}
\begin{tabular}{|c|c|c|c|c|c|c|}
\hline
 \multicolumn{1}{| c |}{\textbf{电压} }&\multicolumn{5}{| c |}{\textbf{时间}}&\multicolumn{1}{| c |}{\textbf{平均}} \\

\hline
$xV$&0&0&0&0&0&0\\
\hline
$xV$&0&0&0&0&0&0\\
\hline
$xV$&0&0&0&0&0&0\\
\hline
$xV$&0&0&0&0&0&0\\
\hline
$xV$&0&0&0&0&0&0\\
\hline
$xV$&0&0&0&0&0&0\\
\hline
$xV$&0&0&0&0&0&0\\
\hline
$xV$&0&0&0&0&0&0\\
\hline
$xV$&0&0&0&0&0&0\\
\hline
$xV$&0&0&0&0&0&0\\
\hline
\end{tabular}
\vspace{10pt}
\end {center}

\subsubsection*{2、数据计算} %原始数据
$$q=ne=\frac{0.927\times 10^{-14}}{[t(1+2.26\times10^{-2}\sqrt{t})]^{3/2}} \cdot \frac{1}{V}$$
所以通过代入计算我们可以得到:
\par 第一组:$q_1=xC,n_1=\frac{q_1}{e_0}\simeq x,e_1=\frac{q_1}{n_1}xC,\eta=x \%$
\par 第二组:$q_2=xC,n_2=\frac{q_2}{e_0}\simeq x,e_2=\frac{q_2}{n_2}xC,\eta=x \%$
\par 第三组:$q_3=xC,n_3=\frac{q_3}{e_0}\simeq x,e_3=\frac{q_3}{n_3}xC,\eta=x \%$

 $$\overline{e}=\frac{\sum_{i=1}^{n=6}e_i}{6}=xC$$
 $$\eta=|\frac{\overline{e}-e_0}{e_0}|=x\%$$
 不确定度计算:
 \par $U_a(e)=xC$
 \par
$\therefore $最终表述为:$e\pm U(e)=(x\pm y)\times 10^{-19}C$

\subsubsection*{3、作图} %原始数据
(请大家发挥想象力自行作图,相信大家的聪明才智……)
\subsection*{六、误差分析}
1、由于油滴质量小,可能出现热运动,布朗运动引入误差。
2、由于时间和电压的范围限定,导致油滴所带电荷数比较集中,从而数据分布比较密集,在作图的时候会产生一定的误差。
3、油滴在上升或者下降的过程中因为温度等因素使质量产生改变,导致误差。


\end{CJK*}
\end{document}
