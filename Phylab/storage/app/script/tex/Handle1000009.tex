\subsection*{实验1.利用多普勒测速仪测量物体通过光电门处的速度}
\begin{center}
\begin{tabular}{|c|c|c|c|c|c|c|c|c|}
	\multicolumn{9}{r}{u=340.0m/s}
	\\\hline
	\multirow{2}{*}{电机速度设置\(\%\)}
	
		&\multicolumn{2}{|c|}{ {{ a }} } 
	
		\\\cline{2-9}
	&靠近&远离&靠近&远离&靠近&远离&靠近&远离
	\\\hline
	多普勒速度\(m/s\)
	
		& {{ a }} 
	
	\\\hline
	误差\(m/s\)
	
		& {{ a }} 
	
	\\\hline
	平均误差\(m/s\)
	
		& \multicolumn{2}{|c|}{ {{ a }} }
	
	\\\hline
\end{tabular}
\end{center}
产生的误差较小,在要求的精度范围内。
产生误差的原因:
\begin{enumerate}
	\itemsep=-5pt
	\item 带动小车的电机转速不稳定。
	\item 光电门测量误差
	\item 温度对声速有影响,从而影响测量结果
\end{enumerate}


\subsection*{实验2.加入温度校正后运动物体速度的测量}
\begin{center}
\begin{tabular}{|c|c|c|c|c|c|c|c|c|}
	\multicolumn{9}{r}{T={{xiuzheng_wendu}}$^{\circ}$C\ u={{xiuzheng_shengsu}}m/s}
	\\\hline
	\multirow{2}{*}{电机速度设置\(\%\)}
	
		&\multicolumn{2}{|c|}{ {{ a }} } 
	
		\\\cline{2-9}
	&靠近&远离&靠近&远离&靠近&远离&靠近&远离
	\\\hline
	多普勒速度\(m/s\)
	
		& {{ a }} 
	
	\\\hline
	误差\(m/s\)
	
		& {{ a }} 
	
	\\\hline
	平均误差\(m/s\)
	
		&\multicolumn{2}{|c|}{ {{ a }} }
	
	\\\hline
\end{tabular}
\end{center}
加入温度校正后,误差比未加入温度校正之前小,说明声速受温度影响。
产生误差的原因:
\begin{enumerate}
	\itemsep=-5pt
		\item 温度在测量过程中发生变化。
\end{enumerate}


\subsection*{实验3.手动测量运动物体通过光电门处的速度}
\begin{center}
\begin{tabular}{|c|c|c|c|c|c|c|c|c|}
	\multicolumn{9}{r}{$f_0$={{shoudong_pinlv}}kHz\ u={{shoudong_shengsu}}m/s}
	\\\hline
	\multirow{2}{*}{电机速度设置\(\%\)}
	
		&\multicolumn{2}{|c|}{ {{ a }} } 
	
		\\\cline{2-9}
	&靠近&远离&靠近&远离&靠近&远离&靠近&远离
	\\\hline
	通过光电门时间\(ms\)
	
		& {{ a }} 
	
	\\\hline
	光电门测得速度\(m/s\)
	
		& {{ a }} 
	
	\\\hline
	多普勒频移\(Hz\)
	
		& {{ a }} 
	
	\\\hline
	多普勒测得速度\(m/s\)
	
		& {{ a }} 
	
	\\\hline
	相对误差
	
		& {{ a }} 
	
	\\\hline
\end{tabular}
\end{center}
产生误差的原因:
\begin{enumerate}
	\itemsep=-5pt
		\item 温度在测量过程中发生变化。
		\item 光电门测量值存在误差。
		\item 多普勒频移的测量不准确。
\end{enumerate}


\subsection*{实验4.环境声速的测量}
\begin{center}
	\begin{tabular}{|c|c|c|}
	\hline
		路径长度\(m\)&传播时间\(s\)&环境声速\(m/s\)
	\\\hline
	1.58&{{shengsu_shijian}}&{{shengsu_shengsu}}
	\\\hline
	\end{tabular}
\end{center}