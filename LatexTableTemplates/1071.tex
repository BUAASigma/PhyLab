\documentclass{article}
\usepackage{CJK}
\usepackage{multirow}
\title{反射法求三棱镜的顶角(1071)}

\begin{document}
\begin{CJK}{GBK}{song}

\maketitle

%反射法求三棱镜的顶角
\begin{tabular}{|c|c|c|c|c|c|c|c|c|}
\hline
\multicolumn{4}{|c|}{反射法求三棱镜的顶角}\\
\hline
次数&\alpha_{1}&\alpha_{2}&\beta_{1}&\beta_{2}&\theta&\theta=((\alpha_{2}-\alpha_{1})+(\beta_{2}-\beta_{1}))/2\\
\hline
1& & & & &0&\multicolumn{3}{c|}{\theta值应该为120左右}\\
\hline
2& & & & &0&\multicolumn{3}{c|}{角度输入形式为x,y,表示x度y分}\\
\hline
3& & & & &0&\multicolumn{3}{c|}{其中x为度数,y为分数}\\
\hline
4& & & & &0&\multicolumn{3}{c|}{}\\
\hline
5& & & & &0&\multicolumn{3}{c|}{}\\
\hline
平均& & & & &0&\multicolumn{3}{c|}{}\\
\hline
\end{tabular}

%最小偏向角法测棱镜的折射率
\begin{tabular}{|c|c|c|c|c|c|c|c|c|}
\hline
\multirow{2}{*}{次数}&
\multicolumn{2}{c|}{入射角}&
\multicolumn{2}{c|}{折射角}&
\multicolumn{3}{c|}{\delta_{min}=((\alpha_{2}-\alpha_{1})+(\beta_{2}-\beta_{1}))/2}\\
\cline{2-3}
 &\alpha_{1}&\beta_{1}&\alpha_{2}&\beta_{2}& &\delta_{min}的值应该为51左右\\
 \hline
 1& & & & &0&\multicolumn{3}{c|}{角度输入形式为x.y,表示x度y分&}\\
 \hline
 2& & & & &0&\multicolumn{3}{c|}{}\\
 \hline
 3& & & & &0&\multicolumn{3}{c|}{}\\
 \hline
 4& & & & &0&\multicolumn{3}{c|}{}\\
 \hline
 4& & & & &0&\multicolumn{3}{c|}{}\\
 \hline
\end {tabular}

%掠入射法测量棱镜折射率

%平板玻璃折射率测量

\end{CJK}{GBK}{song}
\end{document}