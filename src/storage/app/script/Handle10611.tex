\subsection*{实验1.物距像距法测凸透镜焦距}
\subsubsection*{(1)f$<$u$<$2f 成倒立放大的实像}

\begin{center}

\begin{tabular}{|c|c|c|c|c|}
\hline 
光源/mm & 光屏/mm & 凸透镜1/mm & 凸透镜2/mm & 均值/mm \\ 
\hline 



%% i %%

& %% i %%


\\
\hline

\end{tabular}
\vspace{10pt}
\end{center}
\[u = {\mid} {x}_{\text{凸}} - {x}_{\text{光源}} {\mid} - {\delta}\]
\begin{center}
${u}_1 = %% U[0][0] %% mm$			${u}_2 = %% U[0][1] %% mm$			${u}_3 = %% U[0][2] %% mm$
\end{center}
$$v = {\mid} {x}_{\text{屏}} - {x}_{\text{凸}} {\mid}$$
\begin{center}
${v}_1 = %% V[0][0] %% mm$		${v}_2 = %% V[0][1] %%	mm$			${v}_3 = %% V[0][2] %% mm$
\end{center}
$$\because f = \displaystyle\frac{uv}{u+v}$$
\begin{center}
$\therefore {f}_1 = \displaystyle\frac{{u}_1{v}_1}{{u}_1+{v}_1} = %% f[0][0] %% mm$	${f}_2 = %% f[0][1] %% mm$		$ {f}_3 = %% f[0][2] %% mm$
\end{center}
$$\therefore {\bar{f}}_1 = \displaystyle\frac{{f}_1+{f}_2+{f}_3}{3} = %% f[0][3] %% mm$$

\subsubsection*{(2)u=2f 成倒立等大的实像}

\begin{center}

\begin{tabular}{|c|c|c|c|c|}
\hline 
光源/mm & 光屏/mm & 凸透镜1/mm & 凸透镜2/mm & 均值/mm \\ 
\hline 



%% i %%

& %% i %%


\\
\hline

\end{tabular}
\vspace{10pt}
\end{center}
$$u = {\mid} {x}_{\text{凸}} - {x}_{\text{光源}} {\mid} - {\delta}$$
\begin{center}
$ {u}_1= %% U[1][0] %% mm$		${u}_2 = %% U[1][1] %% mm$		${u}_3 = %% U[1][2] %% mm$
\end{center}
$$v = \left | {x}_{\text{屏}} - {x}_{\text{凸}} \right |$$
\begin{center}
${v}_1 = %% V[1][0] %% mm$		${v}_2 = %% V[1][1] %%	mm$		${v}_3 = %% V[1][2] %% mm$
\end{center}
$$\because f = \displaystyle\frac{uv}{u+v}$$
\begin{center}
$\therefore {f}_1 = \displaystyle\frac{{u}_1{v}_1}{{u}_1+{v}_1} = %% f[1][0] %% mm$		$ {f}_2 = %% f[1][1] %% mm	$		$ {f}_3 = %% f[1][2] %% mm $
\end{center}
$$\therefore {\bar{f}}_2 = \displaystyle\frac{{f}_1+{f}_2+{f}_3}{3} = %% f[1][3] %% mm$$

\subsubsection*{(3)u $>$ 2f 成倒立缩小的实像}

\begin{center}

\begin{tabular}{|c|c|c|c|c|}
\hline 
光源/mm & 光屏/mm & 凸透镜1/mm & 凸透镜2/mm & 均值/mm \\ 
\hline 



%% i %%

& %% i %%


\\
\hline

\end{tabular}
\vspace{10pt}
\end{center}
$$u = {\mid} {x}_{\text{凸}} - {x}_{\text{光源}} {\mid} - {\delta}$$
\begin{center}
${u}_1 = %% U[2][0] %% mm$			${u}_2 = %% U[2][1] %% mm$			${u}_3 = %% U[2][2] %% mm$
\end{center}
$$ v = {\mid}{x}_{\text{屏}} - {x}_{\text{凸}}{\mid}$$
\begin{center}
${v}_1 = %% V[2][0] %% mm	$		${v}_2 = %% V[2][1] %%	mm	$		${v}_3 = %% V[2][2] %% mm $
\end{center}
$$\because f = \displaystyle\frac{uv}{u+v}$$
\begin{center}
$\therefore {f}_1 = \displaystyle\frac{{u}_1{v}_1}{{u}_1+{v}_1} = %% f[2][0] %% mm$		$ {f}_2 = %% f[2][1] %% mm$ 		$ {f}_3 = %% f[2][2] %% mm$
\end{center}
$$\therefore {\bar{f}}_3 = \displaystyle\frac{{f}_1+{f}_2+{f}_3}{3} = %% f[2][2] %% mm$$
$$\therefore {\bar{f}} = \displaystyle\frac{\bar{f}_1+\bar{f}_2+\bar{f}_3}{3} = %% Average_f %% mm$$
