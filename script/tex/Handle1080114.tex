\documentclass[11pt,a4paper,oneside]{article}
\usepackage[UTF8,adobefonts]{ctex}

\usepackage{wrapfig}
\usepackage{indentfirst}
\usepackage{amsmath}
\usepackage{float}
\usepackage{ulem}
\usepackage{amssymb}
\usepackage[top=1in,bottom=1in,left=1.25in,right=1.25in]{geometry}

\usepackage{color}
\usepackage{xcolor}

\usepackage{multirow}

\begin{document}
\section*{五、实验数据处理}
\subsection*{实验1.激光双棱镜干涉}
\subsubsection*{(1)原始数据记录}

\begin{center}
\begin{table}[htbp]
\begin{tabular}{|c|c|c|c|c|c|c|c|c|c|c|}
\hline
i & 1 & 2 & 3 & 4 & 5 & 6 & 7 & 8 & 9 & 10
\\
\hline
$x_i/mm$


&%% x %%


\\
\hline
i & 11 & 12 & 13 & 14 & 15 & 16 & 17 & 18 & 19 & 20
\\
\hline
$x_i/mm$


&%% x %%


\\
\hline
\end{tabular}
\end{table}
\begin{table}[htbp]
\begin{tabular}{|c|c|c|c|}
\hline
   & 扩束镜 & 透镜成小像 & 透镜成大像 \\
\hline
X/cm

&%% x %%

\\
\hline
\end{tabular}
\end{table}
\begin{table}[htbp]
\begin{tabular}{|c|c|c|c|c|}
\hline
 & \multicolumn{2}{|c|}{b/mm(小像) }& \multicolumn{2}{|c|}{b'/mm(大像)} \\
\hline
左 & %% DATA_SMALL[0] %% &%% DATA_SMALL[2] %% & %% DATA_BIG[0] %% & %%DATA_BIG[2] %% \\
\hline
右 & %% DATA_SMALL[1] %% &%% DATA_SMALL[3]%% & %%  DATA_BIG[1] %% & %% DATA_BIG[3] %% \\
\hline
\end{tabular}
\end{table}
\end{center}

\subsubsection*{(2)数据处理}
用逐差法计算条纹间距${\Delta}x$:
$$\overline{{\Delta}x} = \displaystyle\frac{\sum\limits_{i=1}^{10}{\mid}x_{i+10}-x_i{\mid}}{10{\times}10} = %% DELTA_X %%mm$$

计算波长$\lambda$:
$$\bar{b} = \displaystyle\frac{b_\text{正}+b_\text{反}}{2} = \displaystyle\frac{ (%% DATA_SMALL[0] %%-%% DATA_SMALL[1]%%)+(%% DATA_SMALL[2] %%-%% DATA_SMALL[3] %%)}{2} = %% B_SMALL %%mm$$

$$\bar{b'} = \displaystyle\frac{b'_\text{正}+b'_\text{反}}{2} = \displaystyle\frac{ (%% DATA_BIG[0] %%-%% DATA_BIG[1]%%)+(%% DATA_BIG[2] %%-%% DATA_BIG[3] %%)}{2} = %% B_BIG %%mm$$

$$ S ={\mid} %% LIGHT_SMALL_BIG[0] %% - %% LIGHT_SMALL_BIG[1] %% {\mid}= %% S_SMALL %%cm$$

$$ S' ={\mid} %% LIGHT_SMALL_BIG[0] %% - %% LIGHT_SMALL_BIG[2] %% {\mid}= %% S_BIG %%cm$$

$$ {\lambda} = \displaystyle\frac{{\Delta}x\sqrt{bb'}}{S+S'} = %% LAMDA_LAB %%nm$$

\subsubsection*{(3)不确定度计算}
${\triangle}x$的不确定度:

$10{\triangle}x$的A类不确定度:
$$u_a(10{\triangle}x) = \sqrt{\displaystyle\frac{\sum\limits_{i=1}^{10} (10{{\triangle}x}_i-{10\overline{{\triangle}x}})^2}{10{\times}(10-1)}} = %% UA_10DELTA_X %%mm$$
$10{\triangle}x$的B类不确定度:
$$u_b({10{\triangle}x})=\displaystyle\frac{\bigtriangleup\text{仪}}{\sqrt{3}}
= \frac{0.01}{2{\times}\sqrt{3}} = 0.00289mm$$
$10{\triangle}x$的不确定度:
$$u(10{\triangle}x)=\sqrt{{u_a(10{\triangle}x}^2)+{u_b(10{\triangle}x}^2)} = %% U_10DELTA_X %%mm$$
${\triangle}x$的不确定度:
$${\therefore}u({\triangle}x) = \displaystyle\frac{u(10{\triangle}x)}{10} = %% U_DELTA_X %%mm$$

$$ \displaystyle\frac{{\Delta}b}{b} = \displaystyle\frac{{\Delta}b'}{b} = 0.025 $$

$b'$的不确定度:
$$ u(b') = \displaystyle\frac{ %% B_SMALL %%{\times}0.025}{\sqrt{3}} =%% U_B1 %%mm$$
b的不确定度:
$$ u(b) = \displaystyle\frac{ %% B_BIG %%{\times}0.025}{\sqrt{3}} =%% U_B2 %%mm$$
S的不确定度:
$${\Delta}S = {\Delta}S' = 0.5cm$$
$$ u(S+S') = \sqrt{2} {\times} 0.289 = 0.409cm$$

不确定度的合成:
$$ \ln{\lambda} = \ln{{\Delta}x}+ \displaystyle\frac{1}{2}(\ln{b}+\ln{b'}) - \ln(S+S')$$

$$\displaystyle\frac{\ln{\lambda}}{\lambda} = \displaystyle\frac{\ln{\Delta}x}{{\Delta}x}+\displaystyle\frac{1}{2}(\displaystyle\frac{\ln{b}}{b}+\displaystyle\frac{\ln{b'}}{b'}) - \displaystyle\frac{\ln{(S+S')}}{S+S'} $$

$$\displaystyle\frac{u({\lambda})}{\lambda} = \sqrt{ [{\frac{u({\Delta}x)}{{\Delta}x}}]^2+\frac{1}{4}[{\frac{u(b)}{b}}]^2 +\frac{1}{4}[{\frac{u(b')}{b'}}]^2 + [\frac{u(S+S')}{S+S'}]^2   } = %% RE_LAMDA %%$$

$$ u({\lambda}) = %% U_LAMDA %% nm$$
最终结果为:
$$ {\lambda}{\pm}{u({\lambda})} = %% RESULT_LAMDA %% {\pm} %% RESULT_U_LAMDA %% nm $$

\end{document}